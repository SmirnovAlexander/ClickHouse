\documentclass[xetex,mathserif,serif]{beamer}

% Language settings.
\usepackage{polyglossia}
\setdefaultlanguage[babelshorthands=true]{russian}

% Setting outer theme.
\useoutertheme{infolines}

% Setting font.
\usepackage{fontspec}
\setmainfont{FreeSans}
\newfontfamily{\russianfonttt}{FreeSans}

% Code highlighting.
\usepackage[outputdir=temp]{minted}
\usepackage{xcolor}

% Images.
\usepackage{graphicx}
\usepackage{animate}

\title{ClickHouse}
\author[Александр Смирнов]{Александр Смирнов}
\date{17.09.2020}

\begin{document}

\begin{frame}
	\titlepage{}
\end{frame}

\begin{frame}
	\frametitle{ClickHouse}

	\begin{itemize}
		\item Распределенная аналитическая столбцовая (column-oriented) СУБД
	\end{itemize}
\end{frame}

\begin{frame}
	\frametitle{Почему column-oriented?}

	\begin{center}
		\animategraphics[
			loop,
			controls,
			width=0.7\linewidth
		]{14}{images/row-oriented/row-oriented-}{0}{86}
	\end{center}

	\begin{itemize}
		\item В обычной row-oriented СУБД данные хранятся по строкам
		\item Хорошо подходит для обычных сценариев
		      \begin{itemize}
			      \item изменить поле у пользователя
		      \end{itemize}
		\item Плохо подходит для задачи аналитики
		      \begin{itemize}
			      \item пусть много столбцов, e.g. время, браузер, модель телефона
			      \item строки — события, e.g. просмотр страницы
			      \item нужен отчет по моделям телефона
		      \end{itemize}
	\end{itemize}
\end{frame}


\begin{frame}
	\frametitle{Почему column-oriented?}

	\begin{center}
		\animategraphics[
			loop,
			controls,
			width=0.7\linewidth
		]{14}{images/column-oriented/column-oriented-}{0}{14}
	\end{center}

	\begin{itemize}
		\item Подходит система где данные хранятся по столбцам
	\end{itemize}
\end{frame}


\begin{frame}
	\frametitle{История}

	\begin{itemize}
		\item Yandex.Metrika
		\item Данные по действиям пользователей на сайтах
            \begin{itemize}
                \item как сохранить данные так, чтобы быстро строить отчеты
                \item изначально предагрегировали данные, но пользователи хотели больше вариативности
            \end{itemize}
		\item Пробовали разные столбцовые субд, не подошло
	\end{itemize}
\end{frame}


\begin{frame}
	\frametitle{Распространение}

	\begin{itemize}
		\item Open-source
		\item Есть своя небольшая ниша
		\item В России
            \begin{itemize}
                \item Яндекс
                \item Mail
                \item СКБ Контур
                \item Rambler
            \end{itemize}
		\item За рубежом
            \begin{itemize}
                \item Cloudfare
                \item Wikimedia
                \item Nvidia
            \end{itemize}
	\end{itemize}
\end{frame}

\begin{frame}
	\frametitle{Преимущества}

	\begin{itemize}
		\item Скорость
		\item Гибкость
            \begin{itemize}
                \item доступны все данные, что хотим то и делаем с ними
            \end{itemize}
		\item Масштабируемость
            \begin{itemize}
                \item распределенное хранение данных
            \end{itemize}
		\item Цена 
            \begin{itemize}
                \item цена разработчиков, не дорого
            \end{itemize}
	\end{itemize}
\end{frame}


\begin{frame}
	\frametitle{Недостатки}

	\begin{itemize}
		\item Ограниченная поддержка JOIN
		\item Нет UPDATE и DELETE
		\item Слабая совместимость со стандартом SQL
	\end{itemize}
\end{frame}



% # как хранятся данные в таблицах

% в первую очередь данные хранятся на одной машинк
% далее добавляется распределенность

% # как хранятся данные в merge tree

\end{document}
