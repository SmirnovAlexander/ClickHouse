\documentclass[xetex,mathserif,serif]{beamer}

% Hyperlinks.
\usepackage{hyperref}
\hypersetup{
    colorlinks=true,
    linkcolor=blue,
    filecolor=magenta,      
    urlcolor=blue,
}

% Language settings.
\usepackage{polyglossia}
\setdefaultlanguage[babelshorthands=true]{russian}

% Setting outer theme.
\useoutertheme{infolines}

% Setting font.
\usepackage{fontspec}
\setmainfont{FreeSans}
\newfontfamily{\russianfonttt}{FreeSans}

% Code highlighting.
\usepackage[outputdir=temp]{minted}
\usepackage{xcolor}

% Images.
\usepackage{graphicx}
\usepackage{animate}

\title{ClickHouse}
\author[Александр Смирнов]{Александр Смирнов}
\date{17.09.2020}

\begin{document}

\begin{frame}
	\titlepage{}
\end{frame}


\begin{frame}
	\frametitle{Сценарии работы с данными}

	\begin{itemize}
		\item Kакие производятся запросы, как часто и в каком соотношении
		\item Сколько читается данных на запросы каждого вида - строк, столбцов, байт
		\item Как соотносятся чтения и обновления данных
		\item Какой рабочий размер данных и насколько локально он используется
		\item Какие требования к дублированию данных и логической целостности
		\item Требования к задержкам на выполнение и пропускной способности запросов каждого вида
	\end{itemize}
\end{frame}


\begin{frame}
    \frametitle{Сценарии работы с данными (2)}

	\begin{itemize}
		\item Не существует системы, одинаково хорошо подходящей под существенно различные сценарии работы
		\item Если система подходит под широкое множество сценариев работы, то при достаточно большой нагрузке, система будет справляться со всеми сценариями работы плохо, или справляться хорошо только с одним из сценариев работы
	\end{itemize}
\end{frame}


\begin{frame}
	\frametitle{OLAP -- интерактивная аналитическая обработка}

	\begin{itemize}
		\item Подавляющее большинство запросов -- на чтение
		\item Данные обновляются достаточно большими пачками (> 1000 строк), а не по одной строке, или не обновляются вообще
		\item Данные добавляются в БД, но не изменяются
		\item При чтении, вынимается достаточно большое количество строк из БД, но только небольшое подмножество столбцов
		\item Таблицы являются широкими, то есть, содержат большое количество столбцов
		\item При выполнении простых запросов, допустимы задержки в районе 50 мс
		\item Результат выполнения запроса существенно меньше исходных данных -- то есть, данные фильтруются или агрегируются, результат выполнения помещается в оперативку на одном сервере
	\end{itemize}
\end{frame}


\begin{frame}
	\frametitle{ClickHouse}

	\begin{itemize}
		\item Распределенная аналитическая столбцовая (column-oriented) СУБД
	\end{itemize}
\end{frame}


\begin{frame}
	\frametitle{Почему column-oriented?}

	\begin{center}
		\animategraphics[
			loop,
			controls,
			width=0.7\linewidth
		]{14}{images/row-oriented/row-oriented-}{0}{86}
	\end{center}

	\begin{itemize}
		\item В обычной row-oriented СУБД данные хранятся по строкам
		\item Хорошо подходит для обычных сценариев
		      \begin{itemize}
			      \item изменить поле у пользователя
		      \end{itemize}
		\item Плохо подходит для задачи аналитики
		      \begin{itemize}
			      \item пусть много столбцов, e.g. время, браузер, модель телефона
			      \item строки — события, e.g. просмотр страницы
			      \item нужен отчет по моделям телефона
		      \end{itemize}
	\end{itemize}
\end{frame}


\begin{frame}
    \frametitle{Почему column-oriented? (2)}

	\begin{center}
		\animategraphics[
			loop,
			controls,
			width=0.7\linewidth
		]{14}{images/column-oriented/column-oriented-}{0}{14}
	\end{center}

	\begin{itemize}
		\item Подходит система где данные хранятся по столбцам
		\item Данные, лежащие по столбцам, лучше сжимаются (однородные)
	\end{itemize}
\end{frame}


\begin{frame}
	\frametitle{Преимущества}

	\begin{itemize}
		\item Скорость
		\item Гибкость
		      \begin{itemize}
			      \item доступны все данные, что хотим то и делаем с ними
		      \end{itemize}
		\item Масштабируемость
		      \begin{itemize}
			      \item распределенное хранение данных
		      \end{itemize}
		\item Цена
		      \begin{itemize}
			      \item цена разработчиков, не дорого
		      \end{itemize}
	\end{itemize}
\end{frame}


\begin{frame}
	\frametitle{Недостатки}

	\begin{itemize}
		\item Ограниченная поддержка JOIN
		      \begin{itemize}
			      \item не больше одного подзапроса
		      \end{itemize}
		\item Нет UPDATE и DELETE
		\item Слабая совместимость со стандартом SQL
	\end{itemize}
\end{frame}


\begin{frame}
	\frametitle{Как хранятся данные в таблицах}

	\begin{itemize}
		\item ClickHouse -- модульная система
		\item Разрабатывается снизу вверх
		      \begin{itemize}
			      \item сначала работает на одной локальной машине
			      \item распределённая система собирается из множества локальных систем
		      \end{itemize}
		\item Поддерживает работу различных Storage Engines
	\end{itemize}
\end{frame}


\begin{frame}
	\frametitle{Индекс БД}

	\begin{itemize}
		\item Объект базы данных, создаваемый с целью повышения производительности поиска данных
		\item Часто реализуется B-деревом
		      \begin{itemize}
			      \item хотим быстро доставать данные по идентификатору пользователя
			      \item отображение из значений ключа (e.g. идентификатор пользователя) в смещения на диске
			      \item B-деревья -- дерево поиска с широкими node-ми
		      \end{itemize}
	\end{itemize}
\end{frame}


\begin{frame}
	\frametitle{MySQL}

	\begin{itemize}
		\item Запись данных
		      \begin{itemize}
			      \item данные поступают по времени
			      \item записываются в конец
			      \item поступают перемешанно со всех сайтов
			      \item записи хранятся друг за другом последовательно
			      \item есть B-дерево, которое эти записи адресует по ключу
		      \end{itemize}
		\item Чтение данных
		      \begin{itemize}
			      \item выбрать данные по одному сайту за период времени
			      \item ключ будет составным: идентификатор сайта, дата
			      \item по дереву вынимаем пачку смещений
			      \item выборка будет размазана по всем данным
		      \end{itemize}
		\item Достоинство: просто
		\item Недостаток: очень долго, данные расположены не оптимально для чтения
	\end{itemize}
\end{frame}


\begin{frame}
	\frametitle{Log-structured merge-tree}

	\begin{itemize}
		\item Заведем clustered index по сайтам, чтобы данные были физически упорядочены
		\item Вместо того, чтобы сразу записывать в файл, будем хранить новые записи в оперативной памяти
		\item Когда оперативная память заканчивается, сливаем записи с записями на диске
		\item Предоставляет быстрый доступ по индексу в условиях частых запросов на вставку
	\end{itemize}
\end{frame}


\begin{frame}
	\frametitle{Векторная обработка запросов}

	\begin{itemize}
		\item Проблема: как обрабатывать сложные запросы
		      \begin{itemize}
			      \item сделать интерпретатор выражений
			      \item в аналитической СУБД нужно быстро фильтровать и агрегировать миллиарды строк
			      \item будет тормозить по CPU
		      \end{itemize}
		\item Решения
		      \begin{itemize}
			      \item кодогенерация для компиляции запроса
			            \begin{itemize}
				            \item компилируем машинный код и применяем для каждой строки
			            \end{itemize}
			      \item векторная обработка запроса
			            \begin{itemize}
				            \item все операции пишутся не для отдельных значений, а для векторов
				            \item не только храним данные по столбцам, но и обрабатываем по столбцам
			            \end{itemize}
		      \end{itemize}
	\end{itemize}
\end{frame}


\begin{frame}
	\frametitle{Benchmark}

	\begin{itemize}
		\item \href{http://gg.gg/m73qs}{Сравнение с другими СУБД}
		\item \href{https://clickhouse.tech/benchmark/hardware/}{Сравнение на разном железе}
	\end{itemize}
\end{frame}


\begin{frame}
	\frametitle{История}

	\begin{itemize}
		\item Yandex.Metrika
		\item Данные по действиям пользователей на сайтах
		      \begin{itemize}
			      \item как сохранить данные так, чтобы быстро строить отчеты
			      \item изначально предагрегировали данные, но пользователи хотели больше вариативности
		      \end{itemize}
		\item Пробовали разные столбцовые субд, не подошло
	\end{itemize}
\end{frame}


\begin{frame}
	\frametitle{Распространение}

	\begin{itemize}
		\item Open-source
		\item Есть своя небольшая ниша
		\item В России
		      \begin{itemize}
			      \item Яндекс
			      \item Mail
			      \item СКБ Контур
			      \item Rambler
		      \end{itemize}
		\item За рубежом
		      \begin{itemize}
			      \item Cloudfare
			      \item Wikimedia
			      \item Nvidia
		      \end{itemize}
	\end{itemize}
\end{frame}


\end{document}
